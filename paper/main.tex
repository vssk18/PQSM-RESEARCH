% main.tex - PQSM Research Paper
\documentclass[conference]{IEEEtran}

\usepackage[T1]{fontenc}
\usepackage{lmodern}
\usepackage{amsmath,amssymb,mathtools}
\usepackage{graphicx}
\usepackage{booktabs}
\usepackage{tabularx}
\usepackage{multirow}
\usepackage{siunitx}
\usepackage{microtype}
\usepackage{enumitem}
\usepackage[nameinlink]{cleveref}
\usepackage{placeins}

\usepackage[font=small,labelsep=space]{caption}
\captionsetup[table]{skip=8pt,belowskip=0pt,aboveskip=6pt,justification=centering}
\captionsetup[figure]{skip=8pt,belowskip=0pt,aboveskip=6pt,justification=centering}

\setlength{\textfloatsep}{10pt plus 2pt minus 2pt}
\setlength{\floatsep}{8pt plus 2pt minus 2pt}
\setlength{\intextsep}{8pt plus 2pt minus 2pt}

\graphicspath{{figs/}}
\numberwithin{equation}{section}
\newcommand{\Th}{\ensuremath{T_{\mathrm{h}}}}
\newcommand{\Ltn}{\ensuremath{L}}
\newcommand{\Loss}{\ensuremath{p}}
\newcommand{\Delivery}{\ensuremath{D}}
\newcommand{\Goodput}{\ensuremath{G}}
\newcommand{\tdec}{\ensuremath{t_{\text{dec}}}}
\newcommand{\Attain}{\ensuremath{\mathcal{A}_{\ge 0.95}}}
\newcommand{\Elasticity}{\ensuremath{\mathcal{E}_{G,p}}}
\newcommand{\explain}[1]{\vspace{3pt}\noindent\textit{Explanation—}#1}


\title{Post-Quantum Secure Messaging over MQTT: End-to-End Evaluation Under Network Impairments}
\author{\IEEEauthorblockN{Varanasi Sai Srinivasa Karthik}
\IEEEauthorblockA{GITAM University, Hyderabad}}

\begin{document}
\maketitle

\begin{abstract}
We present a comprehensive evaluation of post-quantum key encapsulation mechanisms (KEMs) for MQTT messaging under realistic network impairments. Through 4,608 experimental runs covering 256 unique network scenarios, we measure user-visible metrics including handshake time, delivery ratio, decrypt latency, and application goodput.
\end{abstract}

\section{Introduction}
We evaluate post-quantum key encapsulation mechanisms (KEMs) for MQTT under realistic network conditions through 4,608 experimental runs across 256 unique scenarios.

\section{Methodology}
\subsection{Experimental Grid}
We tested 6 KEMs (ML-KEM-512, ML-KEM-768, NTRU-Prime-hrss, BIKE-L1, HQC-128, Classic-McEliece-348864) across:
\begin{itemize}
\item Latency: \{10, 50, 100, 150\} ms
\item Loss: \{0, 1, 5, 10\}\%
\item Payload: \{128, 256, 512, 1024\} bytes
\item Rate: \{1, 2, 5, 10\} Hz
\item Replicates: 3 per configuration
\end{itemize}

Total: $6 \times 4 \times 4 \times 4 \times 4 \times 3 = 4,608$ runs with 768 runs per KEM.

\FloatBarrier
\section{Results}
Analysis of 4,608 experimental runs reveals clear performance stratification among PQC KEMs.

\begin{table}[t]
\centering
\caption{Performance Summary (4,608 Runs)}
\begin{tabularx}{\columnwidth}{@{}l c c c c@{}}
\toprule
\textbf{KEM} & \textbf{Handshake} & \textbf{Delivery} & \textbf{Safe Area} & \textbf{Score} \\
& \textbf{(ms)} & \textbf{Ratio} & & \\
\midrule
NTRU-Prime-hrss & 26.1 & 0.981 & 1.00 & 0.695 \\
ML-KEM-512 & 21.9 & 0.982 & 1.00 & 0.693 \\
ML-KEM-768 & 32.8 & 0.981 & 1.00 & 0.694 \\
BIKE-L1 & 50.5 & 0.974 & 1.00 & 0.690 \\
HQC-128 & 55.1 & 0.972 & 1.00 & 0.691 \\
Classic-McEliece & 278.3 & 0.958 & 1.00 & 0.384 \\
\bottomrule
\end{tabularx}
\end{table}

NTRU-Prime-hrss achieves the highest combined score through balanced performance across all metrics.

\FloatBarrier
\section{Discussion}
Our three-gate decision framework provides clear guidance: NTRU-Prime-hrss offers the best balance, ML-KEM-512 excels in handshake performance, while Classic-McEliece fails practical deployment requirements despite highest security.

\section{Conclusion}
Through 4,608 experimental runs, we demonstrate that transport effects dominate cryptographic costs in typical IoT deployments. Our safe-area scoring and loss elasticity metrics enable operators to make informed KEM selections for their specific network conditions.


\end{document}
